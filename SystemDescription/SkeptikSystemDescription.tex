\documentclass{llncs}

\usepackage{xcolor}
\usepackage{enumitem,amsmath,amssymb}
\usepackage{breakurl}    % used for \url and \burl
\usepackage[linesnumbered,boxed,noline,noend]{algorithm2e}
\def\defaultHypSeparation{\hskip.1in}

\usepackage{tikz}
\usepackage{subfig}
\usepackage{array,booktabs,multirow}
\usepackage{placeins}

\usepackage{logictools}
\usepackage{prooftheory}
\usepackage{comment}
\usepackage{mathenvironments}
\usepackage{drawproof}

\renewcommand{\topfraction}{0.85}
\renewcommand{\textfraction}{0.1}
\renewcommand{\floatpagefraction}{0.75}

\title{The \skeptik Proof Compression System}

\author{
  Joseph Boudou\inst{1}
  \thanks{Supported by the Google Summer of Code 2012 program.}
  \and 
  Andreas Fellner\inst{2}
  \thanks{Supported by the Google Summer of Code 2013 program.}
  \and 
  Bruno Woltzenlogel Paleo\inst{3}
  \thanks{Supported by the Austrian Science Fund, project P24300.}
}

\authorrunning{J.\~Boudou \and A. Fellner \and B.\~Woltzenlogel Paleo}

\institute{
  IRIT, Universit\'e de Toulouse, France \\
  \email{joseph.boudou@matabio.net}
  \and 
  Free University of Bolzano, Italy \\
  \email{fellner.a@gmail.com}
  \and 
  Vienna University of Technology, Austria \\
  \email{bruno@logic.at}
}




\begin{document}

\maketitle


\begin{abstract}
This paper describes \skeptik: a system for compressing and improving proofs obtained by automated reasoning tools. 
\end{abstract}

\setcounter{footnote}{0}


\section{Introduction}



\section{User Interface}



\section{Supported Proof Formats}

\subsection{TraceCheck Format}

\subsection{The \veriT Proof Format}

\subsection{The \skeptik Proof Format}



\section{Available Proof Compression Algorithms}


\subsection{RPI}

\subsection{LU}

\subsection{LUniv}

\subsection{RPI[3]LU and RPI[3]LUniv}

\subsection{RedRec}

\subsection{Split}

\subsection{TautologyElimination}

\subsection{StructuralHashing}

\subsection{DAGification}

\subsection{Subsumption}

\subsection{Pebbling}





\section{Implementation Details}

ToDo: statistics about the code

\subsection{Organization of the Code}

ToDo: brief description of the package structure


\subsection{Data Structures for Formulas}


\subsection{Data Structures for Proofs}


\subsection{}



\section{Conclusions and Future Work}

ToDo: limitation: memory consumption in Scala
underlying symbols are strings



\bibliographystyle{splncs}
\bibliography{biblio}


\end{document}

% vim: tw=100
