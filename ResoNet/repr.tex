\documentclass{article}

\usepackage{mathtools,amssymb,amsthm}

\newtheorem{claim}{Claim}

\newcommand{\parts}[1]{2^{#1}}
\newcommand{\card}[1]{\left| #1 \right|}
\newcommand{\ens}[2]{\left\{ #1 ~ \middle| ~ #2 \right\}}

%TODO: remove unused commands
\newcommand{\varSet}{\mathcal{V}}
\newcommand{\axiomSet}{\mathcal{A}}
\newcommand{\conSet}{\mathcal{C}}
\newcommand{\resoSet}{\mathcal{R}}
\newcommand{\resoNet}{\rho}
\newcommand{\resoOrd}{\prec}
\newcommand{\constep}[1]{\overset{#1}{\smile}}
\newcommand{\connected}{\sim}
\newcommand{\cpath}[2]{\left[ #1 \connected #2 \right]}
\newcommand{\conwith}[2]{\prescript{#1}{}{\connected}^{#2}}
\newcommand{\cpathwith}[4]{\left[ #1 \conwith{#2}{#3} #4 \right]}

\newcommand{\Proof}{\varphi}
\newcommand{\node}{\eta}
\newcommand{\conc}[1]{C\left(#1\right)}
\newcommand{\dual}[1]{\overline{#1}}
\newcommand{\nreach}[3]{#1 \overset{#2}{\nrightarrow} #3}
\newcommand{\treach}[3]{#1 \twoheadrightarrow^{#2} #3}

\newcommand{\reso}[1]{\odot_{#1}}
\newcommand{\transition}[1]{\xrightarrow{#1}}
\newcommand{\ncon}[3]{M_{#3} \left( #1 , #2 \right)}


\begin{document}

\section{Totally shrinkable resolution networks}

Let $\varSet$ be a finite set of variables.

A set $\axiomSet$ of axioms on a set $\varSet$ of variables is a relation between parts of $\varSet$,
ie $\axiomSet \subseteq \parts{\varSet} \times \parts{\varSet}$. In the following it is assumed that
\begin{gather}
  \left(\Gamma, \Delta \right) \in \axiomSet \Rightarrow \Gamma \cap \Delta = \varnothing \\
  (\varnothing, \varnothing) \notin \axiomSet
\end{gather}

A set $\conSet_L$ of left contractions on a set $\axiomSet$ of axioms on $\varSet$ is a relation
between parts of $\axiomSet$ and $\varSet$, ie $\conSet_L \subseteq \parts{\axiomSet} \times
\varSet$, such that
\begin{gather}
  \forall (C_L,\ell) \in \conSet_L ,~ (\Gamma, \Delta) \in C_L \Rightarrow \ell \in \Gamma
    \label{cond:ell_in_left} %\\
%  \forall (C_L,\ell),(C'_L,\ell') \in \conSet_L ,~
%    \ell = \ell' \wedge C_L \cap C'_L \neq \varnothing \Rightarrow C_L = C'_L
\end{gather}
$\conSet_L$ is said to be complete if
\begin{equation}
  \forall (\Gamma, \Delta) \in \axiomSet ,~ \forall \ell \in \Gamma ,~
    \exists (C_L, \ell) \in \conSet_L ,~ (\Gamma, \Delta) \in C_L
\end{equation}
Right contractions are defined similarly, replacing condition \eqref{cond:ell_in_left} by
\begin{equation}
  \forall (C_R,\ell) \in \conSet_R ,~ (\Gamma, \Delta) \in C_R \Rightarrow \ell \in \Delta
\tag{\ref{cond:ell_in_left}'}
\end{equation}

A set $\resoSet$ of resolutions is a relation from a set $\conSet_L$ of left contractions to a set
$\conSet_R$ of right contractions on the same set $\axiomSet$ of axioms, ie $\resoSet \subseteq
\conSet_L \times \conSet_R$, such that
\begin{gather}
  (C_L, \alpha_L) \resoSet (C_R, \alpha_R) \Rightarrow \alpha_L = \alpha_R \\
  \forall C \in \conSet_L ,~ \exists D \in \conSet_R ,~ C \resoSet D
    \label{cond:all_left_resolved} \\
  \forall D \in \conSet_R ,~ \exists C \in \conSet_L ,~ C \resoSet D
    \tag{\ref{cond:all_left_resolved}'}
\end{gather}
For each resolution $((C_L,\alpha),(C_R,\alpha)) \in \resoSet$, $\alpha$ is called the \emph{pivot}
of the resolution.

A \emph{resolution network} is a tuple $\langle \varSet, \axiomSet, \conSet_L, \conSet_R, \resoSet \rangle$
such that
$\varSet$ is a finite set,
$\axiomSet$ is a set of axioms on $\varSet$,
$\conSet_L$ is a set of left contractions on $\axiomSet$,
$\conSet_R$ is a set of right contractions on $\axiomSet$,
$\resoSet$ is a set of resolutions from $\conSet_L$ to $\conSet_R$.

A resolution network $\resoNet = \langle \varSet, \axiomSet, \conSet_L, \conSet_R, \resoSet \rangle$
is said to be \emph{well balanced} iff
$
  \card{\axiomSet} < \card{\conSet_L} + \card{\conSet_R} - \card{\resoSet}
$

A resolution network $\resoNet = \langle \varSet, \axiomSet, \conSet_L, \conSet_R, \resoSet \rangle$
is said to be \emph{initializable} iff $\forall A \in \axiomSet$, $\exists \alpha$ such that
$(\{A\},\alpha) \in \conSet_L \cup \conSet_R$.

The \emph{resolution graph} of 
a resolution network $\resoNet = \langle \varSet, \axiomSet, \conSet_L, \conSet_R, \resoSet \rangle$
is the graph
in which vertices are the axioms in $\axiomSet$ and there is a edge between $A$ and $B$ labeled by a
relation $((C_L,\alpha),(C_R,\alpha)) \in \resoSet$ iff 
$
  \left( A \in C_L \wedge B \in C_R \right) \vee
  \left( A \in C_R \wedge B \in C_L \right)
$.
The resolution network is said to be \emph{connexe} if its resolution graph is.

A connectivity path from $A$ to $B$ is a set of resolutions labeling a path in the
resolution graph going from $A$ to $B$,
traversing each node at most once and in which two consecutive edges are labeled with
resolution having different pivots.
The \emph{connectivity relation} of $\resoNet$ is the relation $\connected \subseteq \axiomSet
\times \axiomSet$ such that $A \connected B$ iff there exists a connectivity path between $A$ and
$B$. The set of connectivity paths from $A$ to $B$ (possibly empty) is denoted by $\cpath{A}{B}$.
More precisely, for any $\alpha, \beta \in \varSet$ and any $A,B \in \axiomSet$, $A
\conwith{\alpha}{\beta} B$ denotes the existence of a connectivity path with the resolution implying
$A$ having $\alpha$ as pivot, and the one implying $B$ having $\beta$ as pivot. Similarly,
$\cpathwith{A}{\alpha}{\beta}{B}$ denotes the set all those paths.

Given a resolution network $\resoNet = \langle \varSet, \axiomSet, \conSet_L, \conSet_R, \resoSet
\rangle$, a total order $\resoOrd$ of $\resoSet$ is said to \emph{prevent tautologies} iff
$\forall A=(\Gamma,\Delta), B=(\Gamma',\Delta') \in \axiomSet$ such that $\card{I} > 1$ with
$I = \left( \Gamma \cap \Delta' \right) \cup \left( \Gamma' \cap \Delta \right)$,
$\forall \alpha \in I$ such that $A \conwith{\alpha}{\alpha} B$,
$\forall P \in \cpathwith{A}{\alpha}{\alpha}{B}$,
$\exists R \in P$ s.t. $\forall \beta \in I \setminus \{\alpha\}$,
$\exists R' = ((C_L,\beta),(C_R,\beta)) \in \resoSet$ such that
$A \in C_L \cup C_R \vee B \in C_L \cup C_R$ and $R' \resoOrd R$.

Given a resolution network $\resoNet = \langle \varSet, \axiomSet, \conSet_L, \conSet_R, \resoSet
\rangle$, a total order $\resoOrd$ of $\resoSet$ is said to \emph{prevent left splits} iff
$\forall (C_L,\alpha) \in \conSet_L$, $\forall A, B \in C_L$ and
$\forall R \in \resoSet$ such that $\exists C_R ,~ R = ((C_L,\alpha),(C_R,\alpha))$, then
$\exists P \in \cpath{A}{B}$ such that $\forall R' \in P$, the pivot of $R'$ is
not $\alpha$ and $R' \resoOrd R$.

Given a resolution network $\resoNet = \langle \varSet, \axiomSet, \conSet_L, \conSet_R, \resoSet
\rangle$, a total order $\resoOrd$ of $\resoSet$ is said to \emph{prevent left merges} iff
$\forall (C_L,\alpha), (C'_L,\alpha) \in \conSet_L$, $C_L \neq C'_L$ implies that
$\forall (A,B) \in C_L \times C'_L$ and
$\forall P \in \cpath{A}{B}$ such that $\alpha$ is not the pivot of any resolution in $P$, then
$\exists R = ((C''_L,\alpha),(C_R,\alpha)) \in \resoSet$ such that
$A \in C''_L \vee B \in C''_L$ and $\forall R' \in P$, $R \resoOrd R'$.

Given a resolution network $\resoNet = \langle \varSet, \axiomSet, \conSet_L, \conSet_R, \resoSet
\rangle$, a total order $\resoOrd$ of $\resoSet$ is said to \emph{preserve left contractions} iff
it prevent left splits and left merges. Similar definitions are given for total orders which prevent
right splits or right merges, or which preserve right contractions.

A resolution network $\langle \varSet, \axiomSet, \conSet_L, \conSet_R, \resoSet \rangle$ is
\emph{shrinkable} iff it is well balanced, initializable, connexe and a total ordering of
$\axiomSet$ which prevents tautologies and preserves left and right contractions exists.

A resolution network $\langle \varSet, \axiomSet, \conSet_L, \conSet_R, \resoSet \rangle$ is
\emph{totally shrinkable} iff it is shrinkable and both $\conSet_L$ and $\conSet_R$ are complete.





\section{Correspondance with usual DAG proofs}

In the following we consider that proofs are DAGified in the sense that in any given proofs two
distinct subproofs which are structuraly identical can not be found.



\subsection{From DAG to TSRN}

Let $\Proof$ be a Proof of unsatifiability of a set $\axiomSet$ of clauses on the set $\varSet$ of
variables. We consider the Proof to be a DAG with vertice labeled with conclusion clause, edges to
be directed from the premises to the resolvent and labeled with the only literal wich appears in the
premise's conclusion but not in the resolvent's. The conclusion of a node $\node$ from $\Proof$ is
denoted $\conc{\node}$ and is represented like the axioms of the previous section.

A node $\node'$ is \emph{reachable without a literal $\ell$} from a node $\node$, denoted by
$\nreach{\node}{\ell}{\node'}$, iff there is a (possibly empty) path in $\Proof$ from $\node$ to
$\node'$ on which no edge is labeled with $\ell$.
A node $\node'$ is \emph{reachable with a terminal literal $\ell$} from another node $\node$, noted
$\treach{\node}{\ell}{\node'}$, iff $\exists \node''$ s.t. $\nreach{\node}{\ell}{\node''}$ and there
is an edge labeled with $\ell$ from $\node''$ to $\node'$.

For each $\alpha \in \varSet$, let $A_\alpha^L = \ens{(\Gamma,\Delta) \in \axiomSet}{\alpha \in \Gamma}$
and $R_\alpha^L \subseteq A_\alpha^L \times A_\alpha^L$ a relation such that
\begin{align*}
  (\Gamma,\Delta) R_\alpha^L (\Gamma', \Delta') \Longleftrightarrow \exists \node,\node',\node'',~
    & \node  \text{ is a leaf and } \conc{\node}  = (\Gamma,  \Delta)  &\wedge \\
    & \node' \text{ is a leaf and } \conc{\node'} = (\Gamma', \Delta') &\wedge \\
    & \nreach{\node} {\dual{\alpha}}{\node''} &\wedge \\
    & \nreach{\node'}{\dual{\alpha}}{\node''} &
\end{align*}
$R_\alpha^L$ is obviously reflexive and symmetric (but not transitive). Nevertheless, some kind of
quotient can be constructed in the following way
\begin{align*}
  P_\alpha^L &=
    \ens{X \subseteq A_\alpha^L}
        {\forall (\Gamma,\Delta),(\Gamma',\Delta') \in X ,~
          (\Gamma,\Delta) R_\alpha^L (\Gamma',\Delta')} \\
  \text{ and }
  Q_\alpha^L &= \ens{X \in P_\alpha^L}{\forall Y \in P_\alpha^L ,~ X \not\subset Y}
\end{align*}
The complete set of left contractions can now be defined as
$$ \conSet_L = \ens{(X,\alpha)}{\alpha \in \varSet \wedge X \in Q_\alpha^L} $$
The complete set $\conSet_R$ of right contractions is defined similarly.

The resolution relation $\resoSet$ between $\conSet_L$ and $\conSet_R$ is defined such that
$(C_L,\alpha) \resoSet (C_R,\alpha)$ iff
$$
  \exists \node ~,
    \left( \forall \node' \text{ s.t. } \conc{\node'} \in C_L ,~
              \treach{\node'}{\dual{\alpha}}{\node} \right) \wedge
    \left( \forall \node' \text{ s.t. } \conc{\node'} \in C_R ,~
              \treach{\node'}{\alpha}{\node} \right)
$$

\begin{claim}
The tuple $\langle \varSet, \axiomSet, \conSet_L, \conSet_R, \resoSet \rangle$ as defined above is a
totally shrinkable resolution network.
\end{claim}



\subsection{From TSRN to DAG}

Given a set $\varSet$ of variables, let $\mathbb{A} = \parts{\varSet} \times \parts{\varSet}$. The
resolution operator $\reso{}$ on $\mathbb{A}$ is the partial function from $\mathbb{A} \times \varSet
\times \mathbb{A}$ to $\mathbb{A}$ defined as
$$
  (\Gamma,\Delta) \reso{\alpha} (\Gamma',\Delta') =
    (\Gamma \cup (\Gamma' \setminus \{\alpha\}),
     (\Delta \setminus \{\alpha\}) \cup \Delta')
  \text{ if } \alpha \in \Delta \wedge \alpha \in \Gamma'
$$

Given a resolution network $\langle \varSet, \axiomSet, \conSet_L, \conSet_R, \resoSet \rangle$, 
the connectivity of $A \in \axiomSet$ w.r.t. a variable $\alpha \in \varSet$ in $\resoSet$, denoted
by $\ncon{A}{\alpha}{\resoSet}$ is the cardinal of
$$\ens{((C_L,\alpha),(C_R,\alpha)) \in \resoSet}{A \in C_L \cup C_R}$$

We define a labeled transition system between resolution network and labeled with resolutions: $
\langle \varSet, \axiomSet, \conSet_L, \conSet_R, \resoSet \rangle \transition{R}
\langle \varSet, \axiomSet', \conSet_L', \conSet_R', \resoSet' \rangle
$ iff
\begin{gather}
R \in \resoSet \\
R = \left( (\{(\Gamma,\Delta)\},\alpha), (\{(\Gamma',\Delta')\},\alpha) \right) \in \resoSet \\
\resoSet' = \resoSet \setminus \{ R \} 
\shortintertext{For the remaining condition it is convenient to define the new axiom}
N = ((\Gamma \cup \Gamma') \setminus \{\alpha\}, (\Delta \cup \Delta') \setminus \{\alpha\})
\notag \\
\forall (C_L,\beta) \in \conSet_L ,~
\begin{cases}
  \text{either}   & C_L = \{A\} \wedge \beta = \alpha \\
  \text{or else } & A \notin C_L \wedge (C_L,\beta) \in \conSet_L' \\
  \text{or else } & \ncon{A}{\alpha}{\resoSet} = 1 \wedge
    (C_L \setminus \{A\} \cup \{N\}, \beta) \in \conSet_L' \\
  \text{or else } & \ncon{A}{\beta}{\resoSet} = 1 \wedge
    (C_L \cup \{N\}, \beta) \in \conSet_L' \\
  \text{or else } & (C_L \setminus \{A\} \cup \{N\}, \beta) \in \conSet_L' \\
  \text{or else } & (C_L, \beta) \in \conSet_L' \\
\end{cases}
%\conSet_L' = \ens{(C_L,\beta) \in \conSet_L}{flu}
%                 {\exists (C_R,\beta) ,~ ((C_L,\beta),(C_R,\beta)) \in \resoSet' \\
%\conSet_L' = \ens{(C_L,\beta) \in \conSet_L}
%                 {\exists (C_R,\beta) ,~ ((C_L,\beta),(C_R,\beta)) \in \resoSet' \\
\end{gather}


\section{Garbage}

Let $\Proof$ be a Proof of unsatifiability of a set $\axiomSet$ of clauses on the set $\varSet$ of
variables. We write $\conc{\node}$ the conclusion of the node $\node$ (using the same representation
as for axioms in the previous section).

The fonction $f$ on nodes of $\Proof$ is defined by induction as \begin{itemize}
  \item if $\node$ is an axiom $\conc{\node} = (\Gamma, \Delta) \in \axiomSet$, then
    $$f(\node) = \langle \varSet, \left\{ (\Gamma, \Delta) \right\},
                        \ens{\left(\left\{(\Gamma, \Delta)\right\}, \ell\right)}{\ell \in \Gamma},
                        \ens{\left(\left\{(\Gamma, \Delta)\right\}, \ell\right)}{\ell \in \Delta},
                        \varnothing \rangle$$
  \item if $\node$ is the resolution on pivot $\ell \in \varSet$ of $\node^L$ and $\node^R$, given
  that
  \begin{align*}
    f(\node^L) &= \langle \varSet, \axiomSet^L, \conSet_L^L, \conSet_R^L, \resoSet^L \rangle \\
    f(\node^R) &= \langle \varSet, \axiomSet^R, \conSet_L^R, \conSet_R^R, \resoSet^R \rangle \\
    \text{ and }
    \conc{\node} &= (\Gamma, \Delta)
  \end{align*}
  and assuming that
  \begin{gather}
    \exists! C^L ,~ (C^L,\ell) \in \conSet_R^L \label{cond:uniq_left_contraction} \\
    \exists! C^R ,~ (C^R,\ell) \in \conSet_L^R \tag{\ref{cond:uniq_left_contraction}'}
  \end{gather}
  we define
\end{itemize}


\end{document}

% vim: tw=100
