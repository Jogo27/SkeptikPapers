\section{Introduction}

$\CERes$ is an ingenious cut-elimination method, invented by Alexander Leitsch and Matthias Baaz \cite{BaazLeitsch1999MethodsofCut-Elimination,BaazLeitsch2000Cut-eliminationandRedundancy-eliminationbyResolution,BaazLeitsch2006Towardsaclausalanalysisofcut-elimination}, that uses resolution proof search to avoid certain kinds of redundancies that affect reductive cut-elimination methods: there are proofs for which reductive cut-elimination methods may require non-elementary many reduction steps and produce non-elementarily large intermediary proofs, while $\CERes$'s more global and search-based approach produces a proof in atomic cut normal form without requiring such expensive intermediary steps \cite{BaazLeitsch2009MethodsofCut-Elimination}. 

However, as discussed in section \ref{sec:Redundancy}, there are also cases where reductive cut-elimination methods can produce short proofs while the proofs generated by $\CERes$ are exponentially larger. Thanks to a simplified description of $\CERes$ in section \ref{sec:CERes}, it becomes evident that the source of redundancy is the naive transformation to clause form that was implicitly used by $\CERes$.

Sections \ref{sec:InferencePermutability} and \ref{sec:StructuralClauseForm} develop two techniques to tame redundancy. The first takes inference permutability into account when performing the clause form transformation, thus avoiding the duplication of literals when disjunctions are distributed over conjunctions. The second proposes the use of structural clause form transformation, which does not cause an exponential blow-up in the formula size in the worst case. These two techniques can be combined, resulting in yet another $\CERes$ variant described in section \ref{sec:Combination}.