\section{Taking Inference Permutability into Account}
\label{sec:InferencePermutability}

Redundancy originates in the distribution of disjunction over conjunction during the clause form transformation of the characteristic formula. As conjunctions and disjunctions in the characteristic formula correspond to binary inferences operating, respectively, on cut-ancestors and end-sequent-ancestors in the proof, the amount of distributions can be reduced by first pre-processing the proof and permuting inferences operating on cut-ancestors downward. The rules for inference permutation are shown in Appendix \ref{appendix:InferencePermutation} and in \cite{Woltzenlogel-Paleo2009A-General-Analysis-of-Cut-Elimination-by-CERes}.

\begin{example}
\label{example:InferencePermutation}
Downward permutation of all inferences operating on cut ancestors transforms proof $\varphi$ from Example \ref{example:CutPertinentStruct} into the proof $\psi$ shown below:
\begin{small}
\begin{prooftree}
			\AXC{$\psi'$} \noLine
			\UIC{$(\hF{A} \wedge \hF{B}) \vee \hF{C} \seq \hA{A} \wedge \hB{B}, \hF{C}$}
						\AXC{$C \seq C$} \RightLabel{$cut_2$}
			\BIC{$(\hF{A} \wedge \hF{B}) \vee \hF{C} \seq \hA{A} \wedge \hB{B}, C$}
					\AXC{$\hC{B} \seq \hC{B} $}
							\AXC{$\hC{A} \seq \hC{A}$} \RightLabel{$\wedge_r$}
						\BIC{$\hC{A}, \hC{B} \seq \hC{B} \wedge \hC{A}$} \RightLabel{$\wedge_l$}
						\UIC{$\hC{A} \wedge \hC{B} \seq \hC{B} \wedge \hC{A}$}  \RightLabel{$cut_1$}
				\BIC{$(\hF{A} \wedge \hF{B}) \vee \hF{C} \seq B \wedge A, C $}
\end{prooftree}
\end{small}
where $\psi'$ is:
\begin{small}
\begin{prooftree}
\AXC{$\hA{A} \seq \hA{A} $} \RightLabel{$w_l$}
\UIC{$\hA{A}, \hA{B} \seq \hA{A}$} \RightLabel{$\wedge_l$}
\UIC{$\hA{A} \wedge \hA{B} \seq \hA{A}$} 
		\AXC{$\hA{C} \seq \hA{C}$} \RightLabel{$\hA{\vee_l}$}  		
	\BIC{$(\hA{A} \wedge \hA{B}) \vee \hA{C}, \seq \hA{A}, \hA{C}$}
				\AXC{$\hB{B} \seq \hB{B}$} \RightLabel{$w_l$}
				\UIC{$\hB{A}, \hB{B} \seq \hB{B}$} \RightLabel{$\wedge_l$}
				\UIC{$\hB{A} \wedge \hB{B} \seq \hB{B}$}
						\AXC{$\hB{C} \seq \hB{C}$} \RightLabel{$\hB{\vee_l}$}  		
					\BIC{$(\hB{A} \wedge \hB{B}) \vee \hB{C} \seq \hB{B}, \hB{C}$}\RightLabel{$\wedge_r$}
			\BIC{$(\hA{A} \wedge \hA{B}) \vee \hA{C}, (\hB{A} \wedge \hB{B}) \vee \hB{C} \seq \hA{A} \wedge \hB{B},\hA{C},\hB{C}$} \RightLabel{$c_l$} 		
			\UIC{$(\hF{A} \wedge \hF{B}) \vee \hF{C} \seq \hA{A} \wedge \hB{B}, \hA{C}, \hB{C}$}
\RightLabel{$c_r$} 		
			\UIC{$(\hF{A} \wedge \hF{B}) \vee \hF{C} \seq \hA{A} \wedge \hB{B}, \hF{C}$}
\end{prooftree}
\end{small}

\noindent
Its characteristic formula is:
$$
\struct{\psi} 
\equiv 
(((A\structtimes C) \structplus (B\structtimes C)) \structplus \structdual{C}) \structplus (\structdual{B} \structtimes \structdual{A})
$$

\noindent
And its characteristic clause set is: 
$$
\clauseset{\psi} \equiv \{ \seq A,C \ \ ; \ \ \seq B,C \ \ ; \ \ C \seq \ \ ; \ \ B, A \seq \ \  \}
$$

\noindent 
Thanks to inference permutations, fewer duplications occur in $\clauseset{\psi}$ than in $\clauseset{\varphi}$.
\hfill\QED
\end{example}


\noindent
However, local proof rewriting such as inference permutation are inefficient. It is typical of reductive cut-elimination methods and it is something that cut-elimination by resolution strives to avoid and improve. Therefore, a method for transforming the characteristic formula into clause form that takes the possibility of inference permutation into account without actually having to perform inference permutations is desirable. Such a method ($\normalizePlusTimesS$) is defined in this section, and it is shown in Lemma \ref{lemma:SwapNormalizeCorrespondence} that every rewriting of the characteristic formula according to $\normalizePlusTimesS$ corresponds to a sequence of inference permutation steps according to the rules in Appendix \ref{appendix:InferencePermutation}. Thus, while $\normalizePlusTimes$ does full distribution of disjunction over conjunction, as if inferences operating on cut-ancestors were always indirectly dependent on the inferences operating on end-sequent ancestors above them, $\normalizePlusTimesS$ does partial distribution when the corresponding inferences are independent and permutable without the need for duplications. For the partial distribution to be possible, the characteristic formula must contain extra information to allow the retrieval of the dependencies between the branching inferences.



\begin{definition}
\label{definition:InferenceOccurrences}
Let $\rho$ be an inference in a proof $\varphi$. Then
$
\occInference{\varphi}{\rho}
$
denotes the set of descendents of formulas that occur in axiom sequents containing ancestors of auxiliary formulas of $\rho$.
\end{definition}


\newcommand{\marked}[1]{#1^*}

\begin{example}
In the proof $\varphi$ below, the formulas belonging to $\occInference{\varphi}{\vee_l}$ have been highlighted in blue:
\begin{prooftree}
\AXC{$\hB{A} \seq \hB{A} $}
		\AXC{$\hB{B} \seq \hB{B}$} \RightLabel{$\wedge_r$}
	\BIC{$\hB{A}, \hB{B} \seq \hB{A} \wedge \hB{B}$} \RightLabel{$\wedge_l$}
	\UIC{$\hB{A} \wedge \hB{B} \seq \hB{A} \wedge \hB{B}$}
				\AXC{$B \seq B $}
						\AXC{$A \seq A$} \RightLabel{$\wedge_r$}
					\BIC{$A, B \seq B \wedge A$} \RightLabel{$\wedge_l$}
					\UIC{$A \wedge B \seq B \wedge A$} \RightLabel{$cut_1$}
			\BIC{$\hB{A} \wedge \hB{B} \seq B \wedge A$}
							\AXC{$\hB{C} \seq \hB{C}$} 
									\AXC{$C \seq C$} \RightLabel{$cut_2$}
								\BIC{$\hB{C} \seq C$} \RightLabel{$\vee_l$}
					\BIC{$(\hB{A} \wedge \hB{B}) \vee \hB{C} \seq B \wedge A, C $} 
\end{prooftree}

\noindent
And below, the formulas belonging to $\occInference{\varphi}{cut_1}$ have been highlighted in blue:
\begin{prooftree}
\AXC{$\hB{A} \seq \hB{A} $}
		\AXC{$\hB{B} \seq \hB{B}$} \RightLabel{$\wedge_r$}
	\BIC{$\hB{A}, \hB{B} \seq \hB{A} \wedge \hB{B}$} \RightLabel{$\wedge_l$}
	\UIC{$\hB{A} \wedge \hB{B} \seq \hB{A} \wedge \hB{B}$}
				\AXC{$\hB{B} \seq \hB{B} $}
						\AXC{$\hB{A} \seq \hB{A}$} \RightLabel{$\wedge_r$}
					\BIC{$\hB{A}, \hB{B} \seq \hB{B} \wedge \hB{A}$} \RightLabel{$\wedge_l$}
					\UIC{$\hB{A} \wedge \hB{B} \seq \hB{B} \wedge \hB{A}$} \RightLabel{$cut_1$}
			\BIC{$\hB{A} \wedge \hB{B} \seq \hB{B} \wedge \hB{A}$}
							\AXC{$C \seq C$} 
									\AXC{$C \seq C$} \RightLabel{$cut_2$}
								\BIC{$C \seq C$} \RightLabel{$\vee_l$}
					\BIC{$(\hB{A} \wedge \hB{B}) \vee C \seq \hB{B} \wedge \hB{A}, C $} 
\end{prooftree}
%\hfill\QED
\end{example}






\begin{definition}[$\normalizePlusTimesS$]
\label{definition:NormalizationPlusTimesSwap}
\index{Struct Normalization!Swapped}
%\newcommand{\hF}[1]{{\color{brickred} #1}}
In the rewriting rules below, let $\rho$ be the inference in $\varphi$ corresponding to $\hF{\structtimes_{\rho}}$. For the rewriting rules to be applicable, 
$\hF{S_{n+1}}, \ldots, \hF{S_{n+m}}$ and $\hF{S}$ must contain at least one formula from $\occInference{\varphi}{\rho}$ each (i.e. there must be an atomic subformula $\hF{A}$ of $\hF{S_{n+k}}$ such that $\hF{A} \in \occInference{\varphi}{\rho}$), and $S_1, \ldots, S_n$ and $S'$ and $S''$ should not contain any formula from $\occInference{\varphi}{\rho}$. Moreover, an innermost rewriting strategy is enforced.
$$
\hF{S} \hF{\structtimes_{\rho}} (S_1 \structplus \ldots \structplus S_n \structplus \hF{S_{n+1}} \structplus \ldots \structplus \hF{S_{n+m}}) \normalizePlusTimesS  S_1 \structplus \ldots \structplus S_n \structplus (\hF{S} \hF{\structtimes_{\rho}} \hF{S_{n+1}}) \structplus \ldots \structplus (\hF{S} \hF{\structtimes_{\rho}} \hF{S_{n+m}})
$$
\begin{multicols}{3}{
$$
\hF{S} \hF{\structplus_{\rho}} S' \normalizePlusTimesS  S'
$$

$$
S' \hF{\structtimes_{\rho}} S'' \normalizePlusTimesS  S'
$$

$$
S' \hF{\structplus_{\rho}} S'' \normalizePlusTimesS  S'
$$
}\end{multicols}

\end{definition}

\begin{definition}[Degenerate Inferences]
An inference $\rho$ in a proof $\varphi$ is \emph{degenerate} when all its 
auxiliary formulas are descendants of main formulas of 
weakening inferences. When only some auxiliary (sub)formulas of $\rho$ are descendants of main formulas of weakening inferences, $\rho$ is \emph{partially degenerate}.
\end{definition}


\begin{remark}
The last three rules in Definition \ref{definition:NormalizationPlusTimesSwap} handle connectives that correspond to (partially) \emph{degenerate inferences}, whose auxiliary formulas are introduced by weakening. These rules are related to downward permutation of weakening inferences, as shown in Lemma \ref{lemma:SwapNormalizeCorrespondence}. Because of the last two rules, $\normalizePlusTimesS$ is not confluent.
\end{remark}


\begin{definition}[Swapped Clause Set]
\label{definition:CutPertinentClauseSetSwap}
A \emph{swapped clause set} $\clausesetSwap{\varphi}{S}$ of a proof $\varphi$ is the $\normalizePlusTimesS$-normal-form $S$ of $\struct{\varphi}$ written in sequent notation.
\end{definition}

\begin{remark}
In cases where $\clausesetSwap{\varphi}{S_1} = \clausesetSwap{\varphi}{S_1}$ for any $\normalizePlusTimesS$-normal-forms $S_1$ and $S_2$ of $\struct{\varphi}$, the unique swapped clause set is denoted simply as $\clausesetSwapUnique{\varphi}$.
\end{remark}


\begin{remark}
Swapped clause sets are very similar to \emph{profiles}, which have been defined in \cite{Hetzl2007CharacteristicClauseSetsandProofTransformations}. In fact, the concept of swapped clause set evolved from attempts to find a simpler explanation for profiles. They differ only on proofs that contain  degenerate inferences. In such cases, swapped clause sets are always smaller \cite{Woltzenlogel-Paleo2009A-General-Analysis-of-Cut-Elimination-by-CERes}.
\end{remark}


\begin{definition}[$\CEResSwap$-normal-form]
$\CEResNFSwap{\varphi}{\delta}$ denotes \emph{$\CEResSwap$ normal form} of the proof $\varphi$ w.r.t. the resolution refutation $\delta$ of any swapped clause set $\clausesetSwap{\varphi}{S}$. It is obtained in the same way as a $\CERes$-normal-form, but using a swapped clause set $\clausesetSwap{\varphi}{S}$ instead of the clause set $\clauseset{\varphi}$.
\end{definition}


\begin{example}
\label{example:PlusTimesSwapNormalization}
Let $\varphi$ be the proof below:
\begin{prooftree}
\AXC{$\hA{A} \seq \hA{A} $}
		\AXC{$\hB{B} \seq \hB{B}$} \RightLabel{$\wedge_r^1$}
	\BIC{$\hA{A}, \hB{B} \seq \hA{A} \wedge \hB{B}$} \RightLabel{$\wedge_l$}
	\UIC{$\hA{A} \wedge \hB{B} \seq \hA{A} \wedge \hB{B}$}
				\AXC{$\hC{B} \seq \hC{B} $}
						\AXC{$\hD{A} \seq \hD{A}$} \RightLabel{$\wedge_r^2$}
					\BIC{$\hD{A}, \hC{B} \seq \hC{B} \wedge \hD{A}$} \RightLabel{$\wedge_l$}
					\UIC{$\hD{A} \wedge \hC{B} \seq \hC{B} \wedge \hD{A}$} \RightLabel{$cut^3$}
			\BIC{$\hA{A} \wedge \hB{B} \seq \hC{B} \wedge \hD{A}$}
							\AXC{$\hE{C} \seq \hE{C}$} 
									\AXC{$\hF{C} \seq \hF{C}$} \RightLabel{$cut^4$}
								\BIC{$\hE{C} \seq \hF{C}$} \RightLabel{$\vee_l^5$}
					\BIC{$(\hA{A} \wedge \hB{B}) \vee \hE{C} \seq \hC{B} \wedge \hD{A}, \hF{C} $} 
\end{prooftree}

\noindent
Its characteristic formula is:
$$
\struct{\varphi} 
\equiv 
((\hA{A} \structplus^1 \hB{B}) \structplus^3 (\hC{\structdual{B}} \structtimes^2 \hD{\structdual{A}}))
\structtimes^5
(\hE{C} \structplus^4 \hF{\structdual{C}})
$$

\noindent
Considering that $\{ \hA{A}, \hB{B}, \hE{C}\} \subset \occInference{\varphi}{\vee_l^5}$ and $\{ \hD{A}, \hC{B}, \hF{C} \} \cap \occInference{\varphi}{\vee_l^5} = \emptyset$, the characteristic formula $\struct{\varphi}$ can be normalized in the two ways shown below:
$$
\begin{array}{rcl}
\struct{\varphi} 
& \equiv &
((\hA{A} \structplus^1 \hB{B}) \structplus^3 (\hC{\structdual{B}} \structtimes^2 \hD{\structdual{A}}))
\structtimes^5
(\hE{C} \structplus^4 \hF{\structdual{C}}) \\
%
& \normalizePlusTimesS &
	((\hA{A} \structplus^1 \hB{B})\structtimes^5 (\hE{C} \structplus^4 \hF{\structdual{C}}))
\structplus^3 
	(\hC{\structdual{B}} \structtimes^2 \hD{\structdual{A}}) \\
%
& \normalizePlusTimesS &
((\hA{A} \structtimes^5
(\hE{C} \structplus^4 \hF{\structdual{C}})) \structplus^1 (\hB{B}\structtimes^5
(\hE{C} \structplus^4 \hF{\structdual{C}}))) \structplus^3 (\hC{\structdual{B}} \structtimes^2 \hD{\structdual{A}}) \\
%
& \normalizePlusTimesS &
(
		(
			(
				\hA{A} 
			\structtimes^5 
				\hE{C}
			) 
		\structplus^4 
			\hF{\structdual{C}}
		) 
	\structplus^1 
		(
			(
				\hB{B}
			\structtimes^5 
				\hE{C}
			) 
		\structplus^4 
			\hF{\structdual{C}}
		)
) 
\structplus^3 
(
	\hC{\structdual{B}} \structtimes^2 \hD{\structdual{A}}
) \\
& \equiv &	
	(
		\hA{A} 
	\structtimes^5 
		\hE{C}
	) 
\structplus 
	\hF{\structdual{C}}	 
\structplus 	
	(
		\hB{B}
	\structtimes^5 
		\hE{C}
	) 
\structplus 
	\hF{\structdual{C}}
\structplus 
(
	\hC{\structdual{B}} \structtimes^2 \hD{\structdual{A}}
) \\
& \equiv & S_1
\end{array}
$$

$$
\begin{array}{rcl}
\struct{\varphi} 
& \equiv &
((\hA{A} \structplus^1 \hB{B}) \structplus^3 (\hC{\structdual{B}} \structtimes^2 \hD{\structdual{A}}))
\structtimes^5
(\hE{C} \structplus^4 \hF{\structdual{C}}) \\
%
& \normalizePlusTimesS &
	((\hA{A} \structplus^1 \hB{B})\structtimes^5 (\hE{C} \structplus^4 \hF{\structdual{C}}))
\structplus^3 
	(\hC{\structdual{B}} \structtimes^2 \hD{\structdual{A}}) \\
%
& \normalizePlusTimesS &
	((((\hA{A} \structplus^1 \hB{B})\structtimes^5 \hE{C}) \structplus^4 \hF{\structdual{C}}))
\structplus^3 
	(\hC{\structdual{B}} \structtimes^2 \hD{\structdual{A}}) \\
%
%
& \normalizePlusTimesS &
	((((\hA{A}\structtimes^5 \hE{C}) \structplus^1 (\hB{B}\structtimes^5 \hE{C})) \structplus^4 \hF{\structdual{C}}))
\structplus^3 
	(\hC{\structdual{B}} \structtimes^2 \hD{\structdual{A}}) \\
%
& \equiv &
	(\hA{A}\structtimes^5 \hE{C}) \structplus (\hB{B}\structtimes^5 \hE{C}) \structplus \hF{\structdual{C}}
\structplus 
	(\hC{\structdual{B}} \structtimes^2 \hD{\structdual{A}}) \\
%
& \equiv &
	S_2 \\
\end{array}
$$

\noindent
The swapped clause sets are:
$$
\clausesetSwap{\varphi}{S_1} = \{\ \ \seq \hA{A}, \hE{C} \ \ ; \ \ \seq \hB{B}, \hE{C} \ \ ;
								\ \ \hF{C} \seq \ \ ; \ \ \hF{C} \seq \ \ ; \ \ \hC{B}, \hD{A} \seq \ \ \}
$$
$$
\clausesetSwap{\varphi}{S_2} = \{\ \ \seq \hA{A}, \hE{C} \ \ ; \ \ \seq \hB{B}, \hE{C} \ \ ;
							 \ \ \hF{C} \seq \ \ ; \ \ \hC{B}, \hD{A} \seq \ \ \}
$$

\noindent
It is interesting to note that $\clausesetSwap{\varphi}{S_1} = \clausesetSwap{\varphi}{S_2}$ (because they are sets). This is not a coincidence. It always occurs when the non-confluence is due to non-degenerated applications of the first rewriting rule. A refutation $\delta$ of $\clausesetSwapUnique{\varphi}$ is shown below:
\begin{small}
\begin{prooftree}
\AXC{$\seq \hA{A}, \hE{C}$}
			\AXC{$\seq \hB{B}, \hE{C}$}
					\AXC{$\hC{B}, \hD{A} \seq$} \RightLabel{$r$}
				\BIC{$\hD{A} \seq \hE{C}$} \RightLabel{$r$}
	\BIC{$\seq \hE{C}, \hE{C}$} \RightLabel{$f_r$}
	\UIC{$\seq \hE{C}$} 
			\AXC{$\hF{C} \seq$} \RightLabel{$r$}
		\BIC{$\seq$}
\end{prooftree}
\end{small}

\noindent
$\CEResNFSwap{\varphi}{\delta}$ is the $\CERes$-normal-form obtained with the swapped clause set:
\begin{scriptsize}
\begin{prooftree}
		\AXC{$ \varphi_0 $} \noLine
		\UIC{$(\hA{A} \wedge \hG{B}) \vee \hE{C}, (\hG{A} \wedge \hB{B}) \vee \hE{C} \seq \hC{B} \wedge \hD{A}, \hE{C}$} 
				\AXC{$\hF{C} \seq \hF{C}$} \RightLabel{$cut$}
			\BIC{$(\hA{A} \wedge \hG{B}) \vee \hE{C}, (\hG{A} \wedge \hB{B}) \vee \hE{C} \seq \hC{B} \wedge \hD{A}, \hF{C}$} \RightLabel{$c_l$}
			\UIC{$(A \wedge B) \vee C \seq \hC{B} \wedge \hD{A}, \hF{C}$}
\end{prooftree}
\end{scriptsize}
where $\varphi_0$ is:
\begin{scriptsize}
\begin{prooftree}
\AXC{$\hA{A} \seq \hA{A}$} \RightLabel{$w_l$}
\UIC{$\hA{A}, \hG{B} \seq \hA{A}$} \RightLabel{$\wedge_l$}
\UIC{$\hA{A} \wedge \hG{B} \seq \hA{A}$}
		\AXC{$\hE{C} \seq \hE{C}$} \RightLabel{$\vee_l$}
	\BIC{$(\hA{A} \wedge \hG{B}) \vee \hE{C} \seq \hA{A}, \hE{C}$} 
				\AXC{$\hB{B} \seq \hB{B}$} \RightLabel{$w_l$}
				\UIC{$\hG{A}, \hB{B} \seq \hB{B}$} \RightLabel{$\wedge_l$}
				\UIC{$\hG{A} \wedge \hB{B} \seq \hB{B}$}
						\AXC{$\hE{C} \seq \hE{C}$} \RightLabel{$\vee_l$}
					\BIC{$(\hG{A} \wedge \hB{B}) \vee \hE{C} \seq \hB{B}, \hE{C}$} 
							\AXC{$\hC{B} \seq \hC{B} $}
									\AXC{$\hD{A} \seq \hD{A}$} \RightLabel{$\wedge_r$}
								\BIC{$\hD{A}, \hC{B} \seq \hC{B} \wedge \hD{A}$} \RightLabel{$cut$}
						\BIC{$(\hG{A} \wedge \hB{B}) \vee \hE{C}, \hD{A} \seq \hC{B} \wedge \hD{A}, \hE{C}$} \RightLabel{$cut$}
		\BIC{$(\hA{A} \wedge \hG{B}) \vee \hE{C}, (\hG{A} \wedge \hB{B}) \vee \hE{C} \seq \hC{B} \wedge \hD{A}, \hE{C}, \hE{C}$} \RightLabel{$c_r$}
		\UIC{$(\hA{A} \wedge \hG{B}) \vee \hE{C}, (\hG{A} \wedge \hB{B}) \vee \hE{C} \seq \hC{B} \wedge \hD{A}, \hE{C}$}
\end{prooftree}
\end{scriptsize}

\noindent
As expected, less redundant clause sets and projections result in a $\CEResNFSwap{\varphi}{\delta}$ significantly smaller than $\CEResNF{\varphi}{\delta}$ shown in Example \ref{example:CEResSONormalForm}. 
\hfill\QED
\end{example}


\begin{lemma}[Correspondence between $\normalizePlusTimesS$ and $\swap$]
\label{lemma:SwapNormalizeCorrespondence}
If $\varphi$ is skolemized and $\struct{\varphi} \normalizePlusTimesS S$, then there exists a proof $\psi$ such that $\varphi \swap^* \psi$ and $\struct{\psi} = S$. 
\end{lemma}
\begin{proof}
In Appendix \ref{sec:Correspondence}.
\end{proof}


\begin{theorem}[Unsatisfiability of the Swapped Clause Set]
\label{theorem:UnsatisfiabilityOfClauseSetSwap}\\
For any skolemized proof $\varphi$ and $\normalizePlusTimesS$-normal-form $S$ of $\struct{\varphi}$, $\clausesetSwap{\varphi}{S}$ is unsatisfiable.
\end{theorem}
\begin{proof}
By Lemma \ref{lemma:SwapNormalizeCorrespondence},
there is $\psi$ such that $\struct{\psi} = S$. Clearly, 
$\clausesetSwap{\varphi}{S} = \clausesetSwapUnique{\psi}$. 
But $\clausesetSwapUnique{\psi} = \clauseset{\psi}$, 
since $S$ is also a $\normalizePlusTimes$-normal-form. 
Therefore, $\clausesetSwap{\varphi}{S} = \clauseset{\psi}$ and, 
by Theorem \ref{theorem:Unsatisfiability} and the fact that 
$\clauseset{\psi}$ is equisatisfiable with $\struct{\psi}$, 
$\clausesetSwap{\varphi}{S}$ is unsatisfiable.
\hfill\QED
\end{proof}


\begin{remark}
It is not possible to prove Theorem \ref{theorem:UnsatisfiabilityOfClauseSetSwap} analogously to the proof of the unsatisfiability of the profile shown in \cite{Hetzl2007CharacteristicClauseSetsandProofTransformations,Hetzl2008ProofProfiles.CharacteristicClauseSetsandProofTransformations}, which is essentially based on the fact that the profile of $\varphi$ subsumes $\clauseset{\varphi}$. Unfortunately, $\clausesetSwap{\varphi}{S}$ does not subsume $\clauseset{\varphi}$ in general. In particular, the subsumption fails when $\varphi$ contains degenerate inferences, in which case $\normalizePlusTimesS$ prunes too much of the characteristic formula and then some clauses of $\clauseset{\varphi}$ are not subsumed by any clause of $\clausesetSwap{\varphi}{S}$.
\end{remark}


\noindent
Although $\CEResSwap$-normal-forms are usually much less redundant than $\CERes$-normal-forms. In the worst case, there is still an exponential blow-up when the characteristic formula is converted to a swapped clause set.

\begin{theorem}[Size of Swapped Clause Set]
\label{theorem:SizeOfSwappedClauseSets}
There exist positive constants $k$ and $k'$ and a sequence of proofs $\varphi_1, \varphi_2, \ldots, \varphi_n, \ldots$ such that $\proofsizeSymbol{\varphi_n} \leq k 4^n$ and $|\clausesetSwapUnique{\varphi_n}| \geq k' 2^{2^n}$.
\end{theorem}

\begin{theorem}[Size of $\CEResSwap$-Normal-Form]
\label{theorem:SizeOfCEResSwapNormalForms}
There exist positive constants $k$ and $k''$ and a sequence of proofs $\varphi_1, \varphi_2, \ldots, \varphi_n, \ldots$ such that:
\begin{itemize}
\item $\proofsizeSymbol{\varphi_n} \leq k 4^n$.
\item $\proofsizeSymbol{\CEResNFSwap{\varphi_n}{\delta_n}} \geq k'' 2^{2^n}$, for any refutation $\delta_n$.
\end{itemize}
\end{theorem}
\begin{proof}
For both theorems above, the same sequence of proofs used in 
Theorem \ref{theorem:SizeOfClauseSets} can be used, 
because $\clauseset{\varphi_k} = \clausesetSwapUnique{\varphi_k}$ for any proof $\varphi_k$ in that sequence.
\hfill\QED
\end{proof}
