\section{Conclusion}
\label{sec:Conclusion}

In this paper, a source of redundancy in cut-elimination with the $\CERes$
method was identified and three variants ($\CEResSwap$, $\CEResD$ and
$\CEResDS$) were developed to succesfully tackle this problem. By using
$\CEResD$ or $\CEResDS$, it is possible to avoid an exponential increase in
the size of the clause set extracted from the proof with cuts. This improves
the efficiency of cut-elimination by resolution and allows it to generate
smaller essentially cut-free proofs.

The redundancy uncovered here is closely linked to the fact that the classical
$\CERes$ method relies heavily on the use of contraction and weakening. This
brings difficulties when applying $\CERes$ for logics that treat these
structural inference rules more restrictively, such as substructural and
intuitionistic logics \cite{WoltzenlogelPaleo20012iCERes}. Therefore,
investigating redundancy sources in $\CERes$ not only is beneficial for
improving the efficiency of $\CERes$ and for reducing the complexity of proofs
it produces, but might also lead to insights on how to broaden the set of logics
for which $\CERes$ is applicable.
